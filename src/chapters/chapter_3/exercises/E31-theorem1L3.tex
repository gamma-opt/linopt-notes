
Prove the following theorem:

\begin{theorem*}[Linear independence and basic solutions] 
	\small
	Consider the constraints $Ax = b$ and $x \geq 0$, and assume that $A$ has $m$ LI rows $M = \braces{1,\dots,m}$. A vector $\overline{x} \in \reals^n$ is a \emph{basic solution} if and only if we have that $A\overline{x} = b$ and there exists indices $B(1), \dots, B(m)$ such that
	\begin{enumerate}
		\item[(1)] The columns $A_{B(1)}, \dots, A_{B(m)}$ are LI
		\item[(2)] If $j \neq B(1), \dots, B(m)$, then $\overline{x}_j = 0$.
	\end{enumerate} 
\end{theorem*}

Note: the theorem is proved in the notes. Use this as an opportunity to revisit the proof carefully, and try to take as many steps without consulting the text as you can. This is a great exercise to help you internalise the proof and its importance in the context of the material. I strongly advise against blindly memorising it, as I suspect you will never (in my courses, at least) be requested to recite the proof literally.